	\chapter{最后:我们自己}
\indent

   从深空的结构到所认知的宇宙边界我们已经遨游了如此遥远了。在这之后,在结束这系列课程致前,我想回归到我们这个主题上来。

   当代物理描绘出一幅世界的伟大壁画。在这物理学家所绘制的壁画中,我们作为感知者决定者,能自由表达自己情感的人类,扮演了一个怎样的角色呢?如果世界是一群转瞬即逝的空间和物质的量子,如同一个空间和基本粒子的巨大拼图,那我们是什么?我们也仅仅是由量子和粒子构成的吗?如果是这样,那我们全部都能证明的,个体存在感和独特的个性,又是从何得来的呢?以及我们的价值,我们的梦想,我们的感情,我们的个体知识又是什么?在这个漫无边界又光芒无限的世界中,我们又是什么呢?

   我甚至不能想象用这简单的篇幅去真正完美回答这样一个深奥的问题。这是一个艰深的问题。在当代科学的巨大图景中,有很多我们暂时无法理解的事物。其中,我们最不能理解的一个就是我们自己。但若是回避或者忽略这个问题,我想,我们会遗漏某些关键性的事物。我已经着手于展示科学之光下的世界是何面貌,以及,作为这个世界的一部分的我们的面貌。

   我们,人类,是第一个也是最重要的,观察世界的主体;也是我所尝试组合的显示照片的集体作者。我们是传递图像、工具、信息和知识的交换(本书就是一个例子)网络中的结点。

   但是我们也是我们所感知的世界的一个整体部分;我们不是外部的观察者。我是置身其中。我们对世界的认知来自其中。我们由相同的原子,和像在山中松树和群星间交换的相同的光信号所组成。

   随着我们知识的增长,我们已经认识到我们只是宇宙的一部分,那么小的一部分。

   几个世纪以来,尤其是在上个世纪,这一点变得越来越明显。我们曾经相信我们是在宇宙中心的星球上,其实我们不是。我们曾认为我们是一个特别的存在,一个独立于动植物的种族,其实发现我们是和身边一切生物共同祖辈的后代。我们和蝴蝶落叶松有着共同的祖先。我们只是唯一一个长大了的孩子,唯一地认识到世界没有像小时候那样围绕着孩子们。他们终会认识到我是众生中的一员。参照他们,参照其他的事物,终将了解我们是谁。

   在德国唯心主义的全盛时期,谢林大概认为人性代表了自然的顶峰,最高点,在此现实意识到了自己。今天,从我们现在对自然世界的知识提供的视点看,谢林的想法只会让我们会心一笑。即使我们与众不同,也只是每个人感到自己不同这般。这仅仅如同每个母亲对于她的子女这般的不同。但是对于自然的其余部分,显然就不是这样了。

   在巨大星河中,我们偏居一隅;在构成现实的无限的不同样式的蔓藤花纹之中,我们只是无数我们这样点缀的其中一个。

   我们所构建的宇宙图景在我们思维空间中与我们俱生。我们能够重建并用浅薄手段能够理解的事物,如同图片,和我们作为一部分的现实世界之间,存在着无数的马赛克筛子:我们的无知,我们感知和智慧的有限。我们自然作为主体,作为特定的主体,这个恰恰相同的条件。

   这些条件,康德设想的是那样普遍——从中推出自欧几里得空间甚至到牛顿力学的自然,而因此注定是先验的。无论如何,它们不是。它们是我们种族精神演化的结果,是在连续的演化中的。我们不仅认识到,也渐渐认识到去改变我们理念框架,以使之适应于我们所认知。并且我们尽管缓慢犹豫,但一直在学习去认识到的,正是我们作为部分的现实世界的自然。我们建构宇宙的图景活在我们心中,在理念的空间中;但是他们也或多或少描述了我们所属的宇宙。我们为了更好地描述这个世界而遵循线索。

   当我们谈起“大爆炸”或者空间结构的时候,我们正在做的不是人们千百年来在篝火讲述自由奇幻的故事的延续。它是别的东西的延续。在一天的第一缕光中,看着羚羊在大草原尘土中留下的痕迹,仔细观察,从现实的细节中剔出,寻求无法直接看见,但可以追寻踪迹的东西。在我们总会犯错因而时刻准备在新踪迹出现时转向的意识中,但也同时在认识到如果我们做得足够好,我们最终会正确并找到我们一直所寻找的猎物。这就是科学的自然。

   在这两类人类活动——发明故事和按图索骥——中的困惑,就是在我们当代文化下展现出的对科学的不解和怀疑。这个分歧是个微妙的分歧:破晓时被猎捕的羚羊还未从在夜时故事会中的羚羊众神中远远剔除。

   二者的边界是互通的。神话滋养着科学,科学滋养着神话。但是,一旦我们发现了可以吃掉的一只羚羊,知识的价值将会留存。

   我们的知识持续的映射着世界。它或多或少,但的的确确反映了我们所栖息的世界。我们自身和世界的这个沟通,不是我们和自然的其余部分间的区分。一切事物都一刻不停地相互作用着,在作用中每个事物都产生着与其作用过的事物的痕迹:在这个意义上一切事物都一直在相互交换着信息。

   一个物理系统对另一个系统的信息对其而言没有任何精神性或者主观性:他仅仅是物理决定的一者和他者间的联系。一滴雨包含了天空中出现云的信息;一束光包含了其发射源的颜色;一个时钟有着一天的信息,风承载了将至的暴风的信息;流感承载了我鼻子易感性的信息;我们细胞中DNA包含了一切我们基因编码的信息(在我和父母相像的意义上);以及我大脑充满我经历所积累的信息。我们思维的原始面目就是一个极端富集的信息。而且它在不断积累,交换并且娓娓道来。

   即使我中心供热系统的恒温器也“感知”并“知晓”我家的大气温度,并且拥有他的信息,可以在够热的时候自己关掉。所以,在温暖并自由决定关闭加热与否上,那恒温器的“感知”和“知晓”与我们的“感知”和“知晓”区别是什么?也“知晓”我的存在吗?到底自然中怎样的信息交换产生了我们和我们的思维?

   问题是开放的,有着无数在讨论之中的解答。我相信这是科学最有趣的前沿。在这里主要的过程就是关于被创造。今天新的工具允许我们观察啊活跃大脑的活动,并且能够视听化地定位高度错综复杂的网络。正如2014年新闻所宣布的首个哺乳动物的大脑结构彻底(中观)详细定位被完成。对结构的数学形式可以对应于意识的主观体验的观点现在正在讨论中。这不仅仅是被哲学家讨论着,也正在被神经学家讨论着。

   举一个奇妙的例子,朱利奥·托诺尼——一位工作在美国的意大利科学家——发展了一个数学理论,被称为“积分信息理论”。它试图定性特征化具备意识的系统的必要结构:作为例子的一个方法是,描述我们在清醒(意识)和无梦睡眠(无意识)间到底发生了什么变化。它仍然还在发展阶段。对于我们意识到底如何形成,我们仍然没有令人信服的,已经建立的答案。但对我而言迷雾渐渐清晰。

   有一个问题,特别是关于我们自己,常常让我们困惑的问题:如果我们的行为仅仅是预定的自然法则,那我们决定的自由意味着什么呢?我们自由感和如同我们现在所了解的,世间万物运行的严谨,这二者之间有没有可能不存在一个矛盾呢?我们内在可能有没有,这样一个东西,让逃离自然的规律,让我们用自由思考的力量去扭曲和偏离自然规律?

   好吧,答案是否。这里没有能让我们逃离自然法则的东西。如果我们内在存在着可以违背自然的某种事物,到现在我们可能早就发现它了。我们内在没有和事物的自然行为存在矛盾之处。整个现代科学——从物理到化学以及从生物到神经科学——只是仅仅证实了这种观察。

   对这种困惑的答案到处都是。当我们说我们是自由的并且我们可能自由是真的,这就意味着我们如何行为是由我们内部,在大脑中发生的事物所决定的,而不仅仅是外部因素。是自由的不意味着我们的行为不受自然法则决定。这只是意味着它受我们大脑中活动的自然法则所决定。

   我们自由决定是由我们大脑中数亿神经元间的,丰富和流动的交互结果所决定:它们是自由的。从这些神经元所能允许和决定的交互的意义上来说是这样的。这可以意味着当我作决定,是“我”真正决定了?是的,当然。因为下面这个问题显得很荒诞。“我”到底能不能做与我整个复杂神经元所决定的不同的事物:这两者,如同荷兰哲学家巴鲁奇·斯宾诺莎在十七世纪所无比清晰理解的一样,是完全相同的。

   单独的“我”和“大脑中的神经元”都是不存在的。它们是相同的东西。一个个体,就是一个进程:复杂,高度集中。

   当我们说人类行为不可预测的时候我们是对的,因为预测太过复杂,尤其是被我们自己预测。我们强烈的内在自由感,如同斯宾诺莎敏锐观察到的,来自于我们关于自生的想法和图景相比于我们内在发生的细节的复杂事物来说是远远地更加粗略的。我们就是我们自己眼中无尽惊奇的源头。

   我们大脑中有数百亿神经元,像星系中的群星那般多的。这些神经元甚至有着天文般更多的连接和可能组合。通过这些组合神经元可以相互作用。我们不可能意识到这全部。“我们”是由这复杂整体组成的进程,而不是其中使我们有意识的一小部分。

   决定的“我”就是自我反思构成(某种意义上仍然不完全清晰但是我们已经开始了解)的“我”;也是在世界中自我表达构成的“我”;也是自身理解为置于世界背景下的一个可变化的观察点的这种想法构成的“我”;也是大脑中处理信息组成表达的宏大结构构成的“我”。当我们已经感受到“这就是我”在做决定的时候,我们就是正确的。如果不是那又是别的谁呢?

   如同斯宾诺莎所维持的,我就是我的身体和发生在我心脑中的事物,也是在于我而言的他们的巨大的不可思议的巨大复杂性。

   涉及到这些篇幅的,我所拥有的科学的世界图景,不会,和我们对自己的感知相不适。它也不会和道德或者心理学术语下我们的思维相不适,也不会和我们情绪和感受相不适。世界是复杂的,我们用不同语言去描述它,每个语言都与我们正在描述的进程相适宜。每个复杂进程可以在不同语言中和在不同层次下被定位和理解。这些丰富的语言交互相织也相互促进,就像这些进程本身一样。在我们对大脑生化学理解之下,对心理的研究变得更加复杂。使我们活跃的热情和情绪滋养着理论物理的研究。

   我们的道德价值,我们的情绪,我们的爱,不再那么缥缈地作为自然的一部分,缥缈地和动物所共享共享或者缥缈地被我们种族百万年来潜移默化的演变所决定。他们只是我们创造的复杂现实世界。我们的现实就是眼泪和欢笑,感恩和无私,忠诚和背叛,过去萦绕我们的,和平静。我们的现实世界有社会,有音乐激发的情绪,由我们共同构建的寻常知识组成的富集的交织的网络。所有的这些只是被我们描述的自同为“自然”的一部分。我们是自然的一个整体部分。我们就是一个有着无穷无尽变量的表达式中的自然。这就是我们从对于世界事物不断增长的认知中所学会的。

   使我们明确成为人类不表示我们和自然的分离;这是自同自然的一部分。这是发生在我们星球上的,在组合的无尽表达下,通过相互影响和部分之间信息和关联的交换,是这样的形式,谁知道有多少和其他的非凡的复杂事物存在着,在无尽的宇宙空间中,大概以我们所不可能想象的形式……有如此多的空间,以至于认为在平凡星系中的偏僻角落中存在某种独一无二特殊的事物,这种想法显得很幼稚。地球上的生命仅仅提供了宇宙中正发生的冰山一角。正是我们的灵魂本身是唯一一个如此小的例证。

   我们是一个自然地被好奇心驱动的种族,有许多同等富于好奇心的种族构成的一群种族(属)中唯一剩下的一个。这群中的其他,种族已经灭绝了;有的,像尼安德特人,很近,大概三千年前灭绝。我们所属的是一群在非洲演化的种族,类似于有层级和好争吵的大猩猩——还甚至更像倭黑猩猩,一种小的安静的,乐观平等,和混种的黑猩猩。一群能够为了探索新世界反复走出非洲的种族。而且他们走的很远:最终远,到巴塔哥尼亚——并且最终远到月球。

   有好奇心不是对抗自然:只是我们的自然就是这样。

   十万年来我们种族离开了非洲。我们可能恰恰被这种好奇心驱使,学会看向甚至更遥远的彼方。在夜晚飞过非洲的航班上,我曾徜徉:这些遥远的走向北方开阔空间的先祖中的一位,可能会抬头望望天,想象着一个遥远的后代会飞去那里,会思考万物的自然,也会仍然被那个恰恰相同的好奇心驱动。

   我相信我们的种族不会延续很久。它不像是由那种东西构成。举个例子那种让乌龟能够延续并或多或少不被改变的存在上亿年。我们属于短命的种属。我们的亲戚都已经灭绝了。何况我们造成了伤害。我们导致的残酷的气候和环境不太可能放过我们。对于地球来说这些就像一些无关紧要的昙花一现,但是我不认为我们能够从它们中毫发无损地逃出生天——尤其是公众和政治观点倾向于忽略这些危险,就像我们在逃避把头埋进沙子里。我们可能是地球上仅存的对我们个人死亡不可避免有着感知的物种。我害怕很快我们就已经成为了唯一的将会看着自己集体灭亡或者自身文明消亡到来的物种。

   正如我们或多或少知道如何去处理个体的死亡,我们应该会处理我们文明的崩塌。这没有多大不同。并且这注定不是第一次发生。玛雅和克里特,包括它们在内的许多其他文明,都已经经历过了。我们生死如同星辰生死,都是个体性的也是集体性的。这就是现实。生对于我们来说是宝贵的。因为它转瞬即逝。并且正如卢克莱修所写:“我们求生若饥,我们求生若渴。”(物性论,III,1084)但是沉浸在创造我们指引我们的自然中,我们不是无家可归的,悬在两个世界,世界的部分但是仅仅部分属于自然的却追求别的事物的物种。不:我们就在家里。

   自然就是我们的家。在自然中就是在家里。这个奇怪的五彩斑斓的和令人令人震惊的世界,我们探索的世界——在这里空间颗粒化,时间不存在,事物无处不在——不是别的正是让我们对真正自我感到奇怪的东西。因为这是唯一我们自然的好奇心所揭示给我们关于的我们住所的东西。关于我们是什么构成的。我们是由构成万物一样的星辰构成的。当我们沉浸在痛苦之中或者当我们体验巨大的快乐的时候,我们不是别的我们只能是:我们世界的一部分。

   卢克莱修表达得很美好:

 \emph{...我们都诞生自同一个天降的种子;}

 \emph{我们中所有人都有同样的父亲,}

 \emph{来自地球,这个喂养我们的母亲,}

 \emph{接受澄澈的雨滴,}

 \emph{从它们中养育繁茂的小麦}

 \emph{和茂密的树木,}

 \emph{和人类,}

 \emph{和野兽的种群,}

 \emph{提供所有滋养一切躯体的食物,}

 \emph{带来美好的生活}

 \emph{并生生不息...}

 \emph{(II,991-7)}



   爱和诚实是我们自然的一部分。渴求了解更多,并继续学习也是我们自然的一部分。我们对世界的知识不断增长。

   这里是我们正在学习的前沿。于此我们燃烧着对知识的渴望。 他们在空间结构最微小的距离中,宇宙的起源中,时间的本质中,黑洞的现象中,以及我们自己思想过程的运作中。 在这里,在我们所知道的边缘,与未知的海洋相接触,照亮了世界的奥秘和美丽。 正是令人叹为观止。


\noindent
