	\chapter{颗粒}
\indent

   在之前那一章被描述的宇宙中,光和事物都是运动的。光是由光子组成的,而光子是爱因斯坦假想出来的微粒。我们看到的每样事物都是由原子构成的,每个原子都包含着原子核和核外电子。每一个原子核又是由紧密连接的质子和中子组成的。质子和中子甚至都是由更小的微粒构成的,这种微粒被美国物理学家默里·盖尔曼称为“夸克”,而这个名词竟出自詹姆斯·乔伊斯的《芬尼根的守灵夜》中看似荒诞的一句话“为马克检阅者王三声夸克!”不管怎样,我们接触的每一件事物都是由电子和一些夸克组成的。

    在质子和中子中把夸克聚集在一起的力是由一种微粒产生的,物理学家没有一丝荒谬之感的称这些微粒为“胶粒”。

    电子,夸克,光子和胶粒都是我们身边空间中每件事物的组成成分。它们都是粒子物理中被研究的基本粒子。除了这些粒子之外,再加上例如聚集在整个宇宙中但和我们几乎没有相互作用的中微子,还有最近在日内瓦欧洲核子研究中心大型强子对撞机中发现的希格斯玻色子。但是它们的种类并不多,事实上少于10中。这些大量的基本材料就像在乐高积木中的砖块一样,组成了我们身边全部的真实的物体。

    这些粒子的性质以及他们是怎么运动的,是被量子力学研究的。这些粒子不像现实生活中的卵石水晶那样真实存在,但它们确实相当于相应领域中的量子,仅仅就像光子是电磁场领域的量子一样。他们是类似于法拉第和麦克斯韦领域中下层基本的激发体,微小移动的小波。他们按照量子力学中奇怪的法则消失和再现,在量子力学中,每一件事物都不是永恒存在的,每一件事物都是从一个作用变成另外一个作用。

    即使我们观察一个很小的空区域,并且在该区域中没有原子,我们仍然可以在某一片刻探测到涌动的粒子。真正的没有东西的虚空是不存在的。就像最平静的海洋中近观都会有微小的波动,所以形成世界的场都容易受到瞬时波动的影响,而且可以想象到它的基本粒子短暂的存在,并被这些运动持续的消灭和产生。

    这使由量子力学和粒子学说所描述的宇宙。我们已经到达了距离牛顿和拉普拉斯的力学世界很远的地方,在他们的世界里,冷石头可以永远的在几何上一成不变的空间中沿着漫长而准确的轨道上运动。量子力学和粒子实验已经告诉我们世界是一个连续的、不停涌动的东西;一个不停出现又消失的短暂的实体。是一系列的如同20世纪60年代嬉皮士世界一样的震动,一个由发生而不是事物组成的世界。

    粒子学说的细节是在20世纪50、60、70年代由那个世纪的像理查德·费曼和默里·盖尔曼一样最伟大的物理学家所逐渐建立的。这些建设工作引导向了基于量子力学的错综复杂的理论,并且有了一个“基本粒子的标准模型(the Standard Model of elementary particles)”这样一个非常浪漫的名字。“标准模型”是在20世纪70年代在进行一系列证实了所有预言的实验后总结出来的。它在2013年发现希格斯波色子后最终确立。

    尽管有一系列成功的实验,这个标准模型从未真正被物理学家们严肃对待。这个理论第一眼看上去破碎而缺乏通盘计划。它由一系列零部件毫无头绪地拼凑而成。若干个领域(确切的说为什么是他们?)通过若干力(为什么是这些力?)相互作用,它们由守恒量(为什么是这些量?)决定并展现出某种对称性(又一次,为什么是这些对称性?)。我们还离简单的的统一的方程,离量子力学很远。

   应用标准模型的方程组做预测,所使用的方法也是十分复杂而混乱的。如果直接使用这些方程,算出的每个被计算量都是无穷大的,结果是没有物理意义的。为了获得有物理意义的结果,不得不设想输入到方程组的常数本身就是无穷大的,来补偿荒谬的结果,使之合理化。这个复杂的和巴洛克风格的过程是在技术上被称为“重整化(renormalization)”。它确实生效了,但是使得那些追寻自然之简洁的人如鲠在喉。继爱因斯坦之后的二十世纪最伟大的科学家、量子力学的伟大建筑师、第一个重要的标准模型方程提出者,保罗·迪拉克,在他生命最后的时光里不断重复强调他对于目前状况的不满,得出“我们还没有解出这个问题”的结论。

   此外,标准模型的另一个显著限制也在近些年显现出来。几乎每一个星系天文学家都是在观察一大片物质云,它通过对恒星的引力而显示自己的存在,它也会偏折光线。我们虽然能观测到它的引力效应,但是由于不能直接观测到它所以不能确定它的组成。为此提出了大量的假说,但没有一个是有效的。我们清楚的是那里一定有东西,但我们不清楚它是什么。现在,我们都把它称作“暗物质”。证据表明,它不能被“标准模型”所描述,否则我们就能看到它了。它是一种原子、中微子、光子等等之外的存在。

   一点也不令人奇怪的是,天上和地上的事物比我们物理或者哲学所能设想到的要多。充满宇宙的无线电波和中微子,我们也是近些年来才相信它们的存在的。如今对于物质世界,标准模型包含了所有理论之精华,它的所有预言也都得到了确认,除了暗物质,它能很好的描述已知宇宙的方方面面,引力也被广义相对论描述为时空的扭曲。

   前面我们已经提出了几种只能被实验所驳倒的理论。19世纪70年代有人提出的一个完善的理论并赋予了它一个专门的名字——SU5。举个例子,这个理论用一个更加简单和优雅的结构替代了标准模型中的无序的等式。这个理论预测一个质子有一个特定的可能性会衰变,转化成电子和夸克。人们建造了大型机械来观察质子的衰变。物理学家们奉献了他们的一生来寻找一个可观察的质子衰变。(你不能一直盯着一个质子看,因为质子的衰变需要很长的时间。应该用成吨的水进行实验并用敏感的传感器来检测是否发生衰变。)但是,唉,至今还没有观察到任何质子衰变的发生。SU5这个美丽的理论,尽管有着十分优雅但它并不被上帝所喜爱。

   这个故事也许正在借着被称作“超对称”的理论重复着。“超对称”理论预测了新一级别的粒子的存在。在我的职业生涯中已经听过很多对这些粒子即将出现充满信心的同事的言论。一天,一个月,一年甚至十年过去了——但是超对称的粒子仍然没有出现。物理学不只是成功者的历史。

   所以,现在我们还要坚持使用标准模型。标准模型也许不是很优雅,但是它在形容我们身边的世界时工作的效果很棒。那么谁知道呢?也许通过更深入的检查它并不是一个缺乏优雅度的模型。也许是我们没有学会从一个正确的角度来看待这个模型。从这个角度来看我们能发现它所隐藏的简易性。目前这是我们关于这个问题所知道的一切:

   许多种类的基础粒子,在存在与不存在之间持续的振动和波动,甚至会蜂拥在一个看似没有任何事物的空间内。无穷多个粒子聚合在一起就像宇宙字母表的文字一样告诉我们星系的,数不胜数的星星的,阳光的,山峰的,森林和成片的谷地的,聚会上年轻人们的笑脸的,镶满星星的夜空的久远历史。

\noindent
